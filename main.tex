\documentclass[11pt,a4paper]{report}
\usepackage[french]{babel}
\usepackage[utf8]{inputenc}
\usepackage[T1]{fontenc}
\usepackage{layout}
\usepackage{titlesec}
\usepackage{lmodern}
\usepackage{enumitem}
\usepackage{amsmath}
\usepackage{empheq}
\usepackage{amssymb}
\usepackage{mathrsfs}
\usepackage{array}
\usepackage{gensymb}
\usepackage{mathenv}
\usepackage{color, colortbl}
%\usepackage[intlimits]{amsmath}
\usepackage[usenames,dvipsnames]{xcolor}
\usepackage{listings}
\usepackage{graphicx}
\usepackage{fancybox}
\usepackage{fourier-orns}



\definecolor{codecustom}{rgb}{0,0.5,0}
\definecolor{codegray}{rgb}{0.7,0.7,0.7}
\definecolor{codepurple}{rgb}{0.58,0,0.82}
\definecolor{backcolour}{rgb}{0.975,0.97,0.975}
\definecolor{custom}{RGB}{216,0,127}
\definecolor{lightgray}{gray}{0.75}
\definecolor{myblue}{rgb}{.8, .8, 1}
\definecolor{redcustom}{RGB}{255,0,0}

\titleformat{\chapter}[hang]{\bf\huge}{\thechapter}{2pc}{}


\usepackage[final]{pdfpages} %inclure un pdf


  \newcommand*\mybluebox[1]{%
    \colorbox{myblue}{\hspace{0.5em}#1\hspace{1em}}}

\lstdefinestyle{mystyle}{
	language=MATLAB,
    backgroundcolor=\color{backcolour},
    commentstyle=\color{codecustom},
    keywordstyle=\color{custom},
    morekeywords={Fe, Te, sin, carre, Liste, ELEMENT},
    numberstyle=\tiny\color{codegray},
    stringstyle=\color{codepurple},
    basicstyle=\footnotesize,
    breakatwhitespace=false,
    breaklines=true,
    captionpos=t,
    keepspaces=true,
    numbers=left,
    numbersep=5pt,
    showspaces=false,
    showstringspaces=false,
    showtabs=false,
    tabsize=2,
    extendedchars=true,
    literate={é}{{\'e}}1 {Ú}{{\`e}}1 {Ã}{{\`a }}1 {ç}{{\c{c}}}1 {œ}{{\oe}}1 {ÃŜ}{{\`u}}1
{É}{{\'E}}1 {È}{{\`E}}1 {À}{{\`A}}1 {Ç}{{\c{C}}}1 {Œ}{{\OE}}1 {Ê}{{\^E}}1
{ê}{{\^e}}1 {î}{{\^i}}1 {ÃŜ}{{\^o}}1 {û}{{\^u}}1 {â}{{\^a}}1,
}

\lstset{style=mystyle}

\frenchbsetup{StandardItemLabels=true, CompactItemize=false, ReduceListSpacing=true}
\setcounter{tocdepth}{4}
\usepackage{caption}
\DeclareCaptionFont{white}{\color{white}}
\DeclareCaptionFormat{listing}{\colorbox{codecustom}{\parbox{\textwidth}{#1#2#3}}}
\captionsetup[lstlisting]{format=listing,labelfont=white,textfont=white}

\renewcommand{\lstlistingname}{Code Extracted}

%%%%%%%% Espace entre paragraphe et Indentation %%%%%%%%%%%%%%%%%%%%

\setlength{\parskip}{0.09cm}
\setlength{\parindent}{0.8cm}

%%%%%%%%%%%%%%%%%%%%%%%%%%%%%%%%%%%%%%%%%%%%%%%%%%%%%%%%%%%%%%%%%%%%


%%%%%%%%%%%%%%%%%%%%%% Dimensions de la page %%%%%%%%%%%%%%%%%%%%%%%%%

\usepackage[top=2cm, bottom=2cm, left=2cm, right=2cm]{geometry}

%%%%%%%%%%%%%%%%%%%%%%%%%%%%%%%%%%%%%%%%%%%%%%%%%%%%%%%%%%%%%%%%%%%%%%

%%%%%%%%%%%%%%%%%%%% Lien Hyperref Sommainre %%%%%%%%%%%%%%%%%%%%%%%%%%


\usepackage{hyperref} % Créer des liens et des signets

%%%%%%%%%%%%%%%%%%%%%%%%%%%%%%%%%%%%%%%%%%%%%%%%%%%%%%%%%%%%%%%%%%%%%%

%%%%%%%%%%% Gestion des en-tête/pieds de page %%%%%%%%%%%%%%%%%%%%%%%%%%%

\usepackage{fancyhdr}
\pagestyle{fancy}

% Permet d'écrire le nom des chapitres et section en minuscules au lieu de majuscules definit par défaut
\renewcommand{\chaptermark}[1]{\markboth{\bsc{\chaptername~\thechapter{} :} #1}{}}
\renewcommand{\sectionmark}[1]{\markright{\thesection{} #1}}
%

\renewcommand{\headrulewidth}{0.4pt} % epaisseur du trait
\fancyhead[C]{}
\fancyhead[L]{\leftmark}
\fancyhead[R]{Guedira}

\renewcommand{\footrulewidth}{0.4pt}
\fancyfoot[C]{\textbf{Page \thepage}}
%\fancyfoot[L]{}
%\fancyhead[C]{}
\fancyhead[R]{\leftmark}
\fancyhead[L]{}

\renewcommand{\footrulewidth}{0.4pt}
\fancyfoot[C]{\textbf{Page \thepage}}
\fancyfoot[L]{\textsc{Ismail Guedira}}
\fancyfoot[R]{\rightmark}

%%%%%%%%%%%%% Option des annexes %%%%%%%%%%%%%%%%%%%%%%%%%%%

\usepackage[toc,page]{appendix}
\renewcommand{\appendixtocname}{Annexes}
\renewcommand{\appendixpagename}{Annexes}
\renewcommand{\appendixname}{Annexes}
%%%%%%%%%%%%%%%%%%%%%%%%%%%%%%%%%%%%%%%%%%%%%%

\newcommand{\gray}{\rowcolor[gray]{.90}}

\usepackage{pifont} %utiliser des signes pour les itemize plutot cool

\usepackage{lscape}	%pouvoir passer en mode paysage



%%%%%%%%%% Change le nom Table en Tableau %%%%%%%%%%

\addto\captionsfrench{\def\tablename{Tableau}}

%@\addto\captionsfrench{\def\figure{Figure}}

\title{Semester Project -- Bandgap Reference Voltage}
\author{Ismail Guedira}
\usepackage{layout}



\begin{document}


%%%%% %%%%%%%%%                       Et si on refaisait la page de garde            %%%%%%%%%%%%%%%%%%%%%%%%%%%%%%

 \makeatletter
   \begin{titlepage}
  \centering
%       {%\includegraphics[height=0.09\textheight]{logo_phelma.png}
%     \hfill \Large \textbf{Grenoble INP - Phelma}}\\
%     \hfill \large \textsc{Physique Électronique Matériaux}\\
%     \hfill  3, Parvis Louis Néel \\
% 	   \hfill  38 000 Grenoble\\
% 
% 	\hspace{3em}
% 
%   %   { \includegraphics[height=0.065\textheight]{logo_Rolls_Royce.jpg} \hfill \Large \textbf{Rolls-Royce}}\\
%   %   \hfill \large \textsc{Civil Nuclear SAS - Instrumentation \& Control }\\
% 	% \hfill 23, Chemin du Vieux Chêne   \\
%   %   \hfill 38 246 Meylan  \\
% 
\hfill \Large   \textsc{\@author} \hfill \\ 
\hfill \normalsize \bfseries Exchange Student EPFL \\
\hfill \normalsize \it Génie Électrique et électronique \\ 
\vfill


 
     \hrule
     \vspace{2em}
     \LARGE \textbf{} \\
     \vspace{2em}
     \Large \textbf{\@title} \\
     \vspace{2em}
     \hrule
    
        
    
%  \hfill   \large $2^{ième}$ Année - \textbf{S}ystèmes \textbf{É}lectroniques \textbf{I}ntégrés \hfill \\
%  \hfill   \textit{ismail.guedira@phelma.grenoble-inp.fr} \hfill \\
% 	\vspace{2em}
% 
%     % \Large \textbf{Maitre de Stage :} 	\hfill	\Large \textsc{Hugo Petit} \\
%    % 										\hfill	\large \textbf{Analog Design Manager} \\
%    % 										\hfill	\textit{hugo.petit.cn@rolls-royce.com} \\
% 
% 
     \vfill
% 
      {\Large \textsc{2014/2015}  \hfill \Large \textsc{\@date}} \\
% 
% 
  \end{titlepage}
\makeatother

%%%%%%%%%%%%%%%%%%%%%%%%%%%%%%%%%%%%%%%%%%%%%%%%%%%%%%%%%%%%%%%%%%%%%%%%%%%%%%%%%%%%%%%%%%%%%%%%%%%%%%%%%%%%%%%%%%%

%\maketitle
\pagenumbering{roman}
\renewcommand{\contentsname}{Sommaire}
\tableofcontents


% \newpage
% \chapter*{Remerciement}
% \addcontentsline{toc}{chapter}{Remerciement}
% 
% REMERCIEMENT
% 
% 
% \chapter*{Introduction}
% \addcontentsline{toc}{chapter}{Introduction}
% 
% INTRODUCTION
% 

\newpage
% 
\addcontentsline{toc}{chapter}{Abstract}
\vspace{3cm}
\textbf{Abstract} \\
\newline
\hspace*{17pt} ABSTRACT

\listoffigures

\listoftables


\newpage

\pagenumbering{arabic}

\chapter{Principle}
\chapter{PTAT Reference}
\section{MOS PTAT}
\section{Bipolar PTAT}
\chapter{CTAT Reference}
\section{VBE = f(T) }
\chapter{Bandgap Voltage}
\section{Current Source}
\section{OTA}
\section{Overall circuits}
\subsection{ Structure 1 ( 3 BJT )}
\subsection{ Structure 2 ( 2 BJT )}
\chapter{Results after Simulation}
\section{Temperature Dependence}
\section{Dependence to VDD - PSRR}
\section{Simulation Post route}

\chapter*{Conclusion}
\addcontentsline{toc}{chapter}{Conclusion}


%%%%%%%%%%%%%%% Début de la gestion de l'annexe %%%%%%%%%%%%%%

\begin{appendices}
\chapter{Evolution of $V_{BE}$ with respect to temperature}

The details of the calculation of the dependence of $V_{BE}$ to temperature. We fisrt devellop the temperature dependence : 

\begin{align}
  V_{BE} & = \frac{k_BT}{q} \cdot ln \left( \frac{I_C}{I_S} \right) \\
  I_S ~   & \alpha ~ \mu ~ k T ~ n_i^2 \\
  \mu ~   & \alpha ~ \mu_0 ~ T^m \\
  n_i^2 ~ & \alpha~ T^3  \cdot  exp\left(\frac{E_g}{kT}\right)
\end{align}

With $\mu$ the mobility of the carriers, $n_i$ the intrinsic carrier concentration of silicon and $E_g~\approx 1.12 eV$ the bandgap energy of silicon.

We introduce a parameter b, which is a proprtionnality factor :

\begin{equation}
  I_S = b T^{4+m} \cdot exp\left(\frac{E_g}{kT}\right)
\end{equation}

Once we have devellop the expression with respect with T, we start to derivate the expression : 

\begin{equation}
    \frac{\partial V_{BE}}{\partial T} = \frac{\partial U_{T}}{\partial T}\cdot ln \left( \frac{I_C}{I_S} \right) + \frac{\partial I_{S}}{\partial T} \cdot \frac{U_T}{I_S}
\end{equation}

\begin{equation}
  \frac{\partial I_{S}}{\partial T} = b(4+m)T^{3+m}\cdot exp\left(\frac{E_g}{kT}\right) + bT^{4+m}\left(exp\left(\frac{E_g}{kT}\right)\right)\cdot \frac{E_g}{k T^{2}} 
\end{equation}

  
\begin{equation}
  \frac{U_T}{I_S} \frac{\partial I_{S}}{\partial T} = (4+m)\frac{U_T}{T} + \frac{E_g}{kT^2}\cdot U_T
\end{equation}

We obtain finally :

\begin{align}
  \frac{\partial V_{BE}}{\partial T} & = \frac{U_T}{T} ln \left( \frac{I_C}{I_S} \right) - (4+m)\frac{U_T}{T} - \frac{E_g}{kT^2}\cdot U_T \\
  \frac{\partial V_{BE}}{\partial T} & = \frac{V_{BE} - (4+m)U_T - E_g/q}{T}
\end{align}



\chapter{ ANNEXE B}
\chapter{ ANNEXE C}
\end{appendices}

%%%%%%%%%%%%%%%%%%%%%%%%%%%%%%%%%%%%%%%%%%%%%%%%%%%%%%%%%%%%



\end{document}
